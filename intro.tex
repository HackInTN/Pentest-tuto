\section{Introduction}

\frame{\tableofcontents[currentsection]}

\begin{frame}{Préambule}
	\begin{itemize}
		\item Ce document est donné à titre indicatif
		\item L'utilisation des outils n'est pas développée, pour l'utilisation utilisez man ou google
		\item La liste des outils n'est pas exaustive, il y a surement beaucoup d'autres outils
		\item Ce document se base sur le livre "Les bases du hacking" de Patrick Engebretson des éditions Pearson.
	\end{itemize}
\end{frame}

\begin{frame}{Quesako}
	\begin{block}{Définition}
	Un test d'intrusion est une méthode d'évaluation de la sécurité d'un système ou d'un réseau informatique.
	La méthode consiste à se faire passer pour un hacker et de voir tout ce qui n'est pas sécurisé afin d'appliquer des correctifs.
	\end{block}

	\begin{block}{Différents types}
		\begin{itemize}
			\item WhiteBox
			\item GreyBox
			\item BlackBox
			\item Red Team
		\end{itemize}
	\end{block}
\end{frame}


\begin{frame}{Droit}
	\begin{alertblock}{Autorisation}
		Pour faire un test d'intrusion vous devez obligatoirement avoir un contrat spéculant ce dont vous avez le droit de faire. Vous devez spécifier le périmiètre de votre test, les sites/machines visées, utilisation de rootkit, ...
	\end{alertblock}

	\begin{block}{HackInTN}
		Pour ce qui est du Pentest à l'école, il est seulement autorisé d'en faire sur le serveur du club via le VPN. Vous ne devez en aucun cas faire de scan sur le réseau de l'école (et de vos voisins (et de tout le reste)).
	\end{block}
\end{frame}

\begin{frame}{Déroulement}
	\begin{block}{}
		\begin{itemize}
			\item Reconnaissance
			\item Scan
			\item Exploitation
			\item Post Exploitation
		\end{itemize}
	\end{block}
\end{frame}