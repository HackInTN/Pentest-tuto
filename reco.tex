\section{Reconnaissance}

\frame{\tableofcontents[currentsection]}

\begin{frame}{But}
	\begin{block}{Récupération d'informations sur}
		\begin{itemize}
			\item Le personnel
			\item La société
			\item La structure interne
			\item Les machines
			\item Les services
		\end{itemize}
	\end{block}
\end{frame}


\begin{frame}{Collecte d'informations}
	\only<1>{
		\begin{block}{HTTrack}
			\begin{itemize}
				\item Clone d'un site web
				\item Examiner le site
				\item Collecte de numéros, e-mails, ...
			\end{itemize}
		\end{block}
	}

	\only<2>{
		\begin{block}{Google,...}
			\begin{itemize}
				\item Indexe tout et n'importe quoi
				\item Opérateurs $\rightarrow$ nom:terme\\
					$\rightarrow$ site, intitle, allintitle, inurl, filetype, cache, ...
				\item Exemple : inurl:admin site:telecomnancy.eu
			\end{itemize}
		\end{block}	
	}

	\only<3>{
		\begin{block}{The Harvester}
			\begin{itemize}
				\item Récupération de sous domaines et e-mails
				\item Trouver des schémas de génération d'e-mail
				\item Choix de la source d'information
			\end{itemize}
		\end{block}
	}

\end{frame}



\begin{frame}{Whois, Netcraft, Hosts}
	\only<1>{
		\begin{block}{Whois}
			\begin{itemize}
				\item Informations sur le propriétaire du domaine
				\item Informations de localisation
			\end{itemize}
		\end{block}
	}

	\only<2>{
		\begin{block}{NetCraft}
			\begin{itemize}
				\item Moteur de recherche
				\item Informations supplémentaires\\
					$\rightarrow$ IP, OS, versions, DNS? ...

			\end{itemize}
		\end{block}
	}
	
	\only<3>{
		\begin{block}{Hosts}
			\begin{itemize}
				\item Récupération d'adresse IP\\
				\textit{(Modèle OSI : couche 3)}
			\end{itemize}		
		\end{block}
	}
\end{frame}



\begin{frame}{DNS}
	\only<1>{
		\begin{block}{DNS}
			\begin{itemize}
				\item Premières cibles
				\item Plan du réseau
				\item Mal protégés/configurés
			\end{itemize}
		\end{block}
	}

	\only<2>{
		\begin{block}{NSlookup}
			\begin{itemize}
				\item Obtenir des enregistrements
				\item Exemple : \\
					\$ nslookup\\
					> serveur X.X.X.X\\
					> set type=any\\
					> domaine.com\\
					Server : Y.Y.Y.Y\\
					Address : Y.Y.Y.Y\#53
				\item \textit{(Modèle OSI : couche 3)}	
			\end{itemize}
		\end{block}
	}

	\only<3>{
		\begin{block}{dig}
			\begin{itemize}
				\item Extraire de l'information
				\item Transfert de zone
				\item Exemple : dig @X.X.X.X domaine.com -t AXFR
				\item Limites $\rightarrow$ Fierce
				\item \textit{(Modèle OSI : couche 3)}
			\end{itemize}
		\end{block}
	}
\end{frame}



\begin{frame}{Messagerie}
	\begin{block}{Messagerie}
		\begin{itemize}
			\item Serveurs internes
			\item Messages automatiques
			\item Messages d'erreurs\\
				$\rightarrow$ Envoie de .bat, .exe, ...
		\end{itemize}	
	\end{block}	
\end{frame}



\begin{frame}{Social Engineering}
	\begin{block}{En attendant une futur présentation}
		\begin{itemize}
			\item Utiliser la naïveté des personnes
			\item Utiliser l'usurpation d'identité
			\item Jouer un rôle avec les cibles
			\item Et plus encore...
		\end{itemize}
	\end{block}
\end{frame}

\begin{frame}{Etat actuel}
	\begin{itemize}
		\item Liste d'URL
		\item Liste d'adresses IP
		\item Liste d'adresses e-mail
		\item Version des équipements accessibles directement
	\end{itemize}
\end{frame}