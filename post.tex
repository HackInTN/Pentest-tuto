\section{Post Exploitation}

\frame{\tableofcontents[currentsection]}

\begin{frame}{But}
	\begin{block}{}
		\begin{itemize}
			\item Installer une porte dérobée
			\item Garder un shell distant disponible même après un patch
			\item Ex-filtration de données
			\item Pivoter
		\end{itemize}
	\end{block}
\end{frame}

\begin{frame}{Netcat et autres}
	\begin{block}{Principes}
		\begin{itemize}
			\item Mettre en place une connexion
			\item Simple d'utilisation
			\item Cryptcat pour une liaison chiffrée
		\end{itemize}
	\end{block}

	\begin{block}{Cas d'utilisations}
		\begin{itemize}
			\item Simple connexion
			\item Envoie de fichiers
			\item Scan de ports
			\item Serveur web léger
		\end{itemize}
	\end{block}
\end{frame}

\begin{frame}{Rootkit}
	\begin{block}{Définition}
		Un rootkit est un ensemble de techniques mises en oeuvre par un ou plusieurs logiciels, dont le but est d'obtenir et de pérenniser un accès (généralement non autorisé) à un ordinateur de la manière la plus furtive possible
	\end{block}

	\begin{block}{Principes}
		\begin{itemize}
			\item Se cacher dans un fichier
			\item Se cacher dans un processus
			\item Keylogger
			\item ...
		\end{itemize}
	\end{block}
\end{frame}

\begin{frame}{Meterpreter}
	\begin{block}{Un couteau suisse}
		\begin{itemize}
			\item Manipulation de fichier
			\item Manipulation d'utilisateurs
			\item Exécution de programme
			\item KeyLogger
			\item Migration dans des processus
			\item Gestion des processus
			\item Pivoter
			\item ...
		\end{itemize}
	\end{block}
\end{frame}

\begin{frame}{Etat actuel}
	\begin{itemize}
		\item Accès à toute la machine
		\item Machine de pivot pour recommencer sur un réseau privé
		\item La fin du pentest, ou presque
	\end{itemize}
\end{frame}